% CUE Talk

% Colors
\usecolors[crayola]
\setupbackgrounds[page][background=color,backgroundcolor=black]
\setupcolors[textcolor=cyan]

% Page Setup
\setuppapersize[S6]
\setuppagenumbering[state=stop]
\setuptolerance[horizontal,verytolerant,stretch] 
\setupitemize[inbetween=] % Avoid ridiculously large vertical spaces between bullet points.

% Columns
\definecolumnset[TwoColumns][n=2, separator=rule]

% Images
\setupexternalfigures[
  location={local,global},
  directory={_images},
  ]

% Links
\setupinteraction[state=start, color=SkyBlue, style=\tf]

% Fonts
\switchtobodyfont[gentium,20pt]

% Font for code formatting
\definetyping[code][
  numbering=line,
  bodyfont=modern,tt,small,
  ]


% Define a "slide" head (each on own page).
\definehead[slide][subject]
\setuphead[slide][page=yes, style=\bfb]

% Big, "Takahashi Method" slide.
\define[1]\slideBig{
  \startslide
    \startalign[middle]
    \scale[factor=fit]{#1}
    \stopalign
  \stopslide
}



% CONTENT
\starttext

% TITLE SLIDE
\startstandardmakeup

  \title{\bf \scale[wfactor=max]{Exploring the CUE Continuum}}

  \startcolor[white]

  {\tfx \it Defining, Generating, and Validating Data with CUE and Go}

  {\tfxx John Gosset \| 2022-07-07 \| Golang Montreal}

  \stopcolor
\stopstandardmakeup

%\setupfootertexts[{\color[white]{\tfxx \midaligned{\hfill \llap{John Gosset}}}}]
%\setupfooter[state=start]

% BEGINNING
\startcolor[lightorange]

  \startslide[title=Overview]
    \startitemize
    \item Audience: Curious Go Devs, the JSON/YAML-demoralized
    \item Goal: Share my interest in CUE+Go, and maybe pique yours!
    \item Non-Goal: Comprehensive overview of CUE
    \item Topics:
      \startitemize
      \item Background
      \item CUE Basics
      \item CUE via Go
      \item Going Further
      \stopitemize
    \stopitemize
  \stopslide



\startslide[title=The Landscape]
  \startitemize
  \item Configuration Languages (\type{.rc}, \type{.ini}, \type{.env}...)
  \item Serialization Languages (JSON, YAML, TOML)
  \item Markup Languages (XML)
  \item --> Structured Data
  \stopitemize
\stopslide

\slideBig{I Just Want to Code!}

\page
  \midaligned{\externalfigure[cue.png]}

  \midaligned{\quotation{Configure, Unify, Execute}}

  \midaligned{\goto{cuelang.org}[url(https://cuelang.org)]}
\page

\startslide[title=CUE Use Cases]
  \startitemize
  \item Configuration
  \item Data Validation
  \item Schema Definition
  \item Code Generation and Extraction
  \item Querying (WIP)
  \item Scripting
  \stopitemize
\stopslide

\startslide[title=About CUE]
  \startitemize
  \item Created by Marcel von Lohuizen (ex-Google, Go Team)
  \item 15 years experience on GCL
  \item Open Source
  \item NON-turing complete
  \item Superset of JSON
  \item Big Idea: Types are Values
  \item Main implementation in Go
  \item Heavily Go-inspired module system
  \stopitemize
\stopslide


\page
  \midaligned{\externalfigure[marcel.png][height=8cm]}
\page

\page
\quotation{A lattice is a partially ordered set, in which every two elements have a unique least upper bound (join) and greatest lower bound (meet). By definition this means there is always a single root (top) and a single leaf (bottom).}

-- \it The Logic of CUE
\page


\page
  \midaligned{\externalfigure[lattice1.png][height=8cm]}
\page

\page
  \midaligned{\externalfigure[lattice2.png][height=8cm]}
\page

\startslide[title=CUE in the Wild]

  \startitemize
    \tfx
  \item \goto{Dagger}[url(https://dagger.io/)]: Portable DevKit for CI/CD Pipelines
  \item \goto{Kubevela}[url(https://kubevela.io/)]: Deployment Plans as Workflow
  \item \goto{grafana/thema}[url(https://github.com/grafana/thema)]: CUE-based framework for portable, evolvable schema (WIP)
  \item \goto{nixago}[url(https://nix-community.github.io/nixago/engines/cue.html)]: Generate config files using Nix (CUE engine)
  \item \goto{cubectl}[url(https://github.com/cuebernetes/cuebectl)]: Truly declarative kubernetes manifests via cuelang
  \item \goto{cue-flux-controller}[url(https://github.com/phoban01/cue-flux-controller)]: A Kubernetes controller for CUE via Flux
  \stopitemize
\stopslide

\page
  \midaligned{\externalfigure[solomon.png]}
\page




\stopcolor

% MIDDLE
\startcolor[lightgreen]

\startslide[title=CUE Basics]
  Via \goto{cuetorials.com}[url(https://cuetorials.com)]

  \startitemize
  \item \goto{JSON Superset}[url(https://cuelang.org/play/?id=qhYIy7ED_zn#cue@export@cue)]
  \item \goto{Definitions}[url(https://cuelang.org/play/?id=-M1_qizIclD#cue@export@cue)]
  \item \goto{Conjunctions}[url(https://cuelang.org/play/?id=7SR677XCUc0#cue@export@cue)]
  \item \goto{Disjunctions}[url(https://cuelang.org/play/?id=3PJ3ka_unU7#cue@export@cue)]
  \item \goto{Defaults and Optionals}[url(https://cuelang.org/play/?id=eakbnQa2zzC#cue@export@cue)]
  \item \goto{Building Up Values}[url(https://cuelang.org/play/?id=RtAWAMD0x1i#cue@export@cue)]
  \stopitemize
\stopslide

\startslide[title=CUE via Go]
  \startitemize
  \item Load CUE code into CUE values
  \item Print CUE values with various options
  \item Extract CUE values, loop over fields and lists
  \item Extract and work with CUE attributes
  \item Unify and validate CUE and Go values
  \item Decode to, and encode from, CUE and Go values
  \item Validate, constrain, and complete Go values
  \stopitemize
\stopslide

\stopcolor

% END
\startcolor[lightblue]

\startslide[title=Alternatives]
  \startitemize
  \item XML
  \item \goto{JSON}[url(https://www.json.org/)]
  \item \goto{YAML}[url(https://yaml.org/)]
  \item \goto{Jsonnet}[url(https://jsonnet.org/)]
  \item \goto{Dhall}[url(https://dhall-lang.org/)]
  \item \goto{Nickel}[url(https://nickel-lang.org/)]
  \item \goto{Terraform/HCL}[url(https://www.terraform.io/language)]
  \item \goto{Pulumi}[url(https://www.pulumi.com/)]
  \item ...
  \stopitemize
\stopslide

\startslide[title=CUE Pros and Cons]
  \startitemize
  \item Pros

  \startitemize
  \item Mathmatically elegant big idea (value lattice) distinguishes from the pack
  \item Excellent interop w/ Go
  \item Familiar for Go devs
  \item Emerging interest
  \stopitemize

  \item Cons
  \startitemize
  \item Pace of dev could be faster
  \item Some features missing/roadmap (pkg management, query, testing, LSP)
  \stopitemize

  \stopitemize
\stopslide

\startslide[title=Next Steps for the Curious]
  \startitemize
  \item \goto{cuelang.org/docs}[url(https://cuelang.org/docs)]
  \item \goto{cuelang.org/play}[url(https://cuelang.org/play)]
  \item \goto{cuetorials.com}[url(https://cuetorials.com/)]
  \item \goto{cuetorials.com/go-api}[url(https://cuetorials.com/go-api/)]
  \item \goto{dagger.io: What is CUE?}[url(https://docs.dagger.io/1215/what-is-cue/)]
  \item \goto{dagger.io: CUE Package Coding Style}[url(https://docs.dagger.io/1226/coding-style/)]
  \item \goto{The Configuration Complexity Curse}[url(https://blog.cedriccharly.com/post/20191109-the-configuration-complexity-curse/)]
  \stopitemize
\stopslide

\startslide[title=Summary]
  \startitemize
    \item Background
    \item CUE Basics
    \item CUE via Go
    \item Going Further
  \stopitemize
\stopslide


\page

\midaligned{\externalfigure[q.jpg][height=5cm]}

\quotation{For that one fraction of a second you were open to options you never considered. That is the exploration that awaits you.
  Not mapping stars and studying nebulae, but charting the unknown possibilities of existance!}

\par

-- \it John de Lancie speaking as \quote{Q}

\page

\slideBig{Thank You!}

\stopcolor

\stoptext
